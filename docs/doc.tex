\documentclass[12pt]{report}
\usepackage[utf8]{inputenc}
\usepackage[czech]{babel}
\usepackage{hyperref}
\usepackage{graphicx} 
\usepackage{float}
\title{Logické řízení - řízení kotle na ohřev vody}
\author{Jan Kohlíček}

\begin{document}

\begin{titlepage}
\begin{flushleft} 
{\includegraphics[width=.5\textwidth]{./images/fav_logo.jpg}\\[3cm]}
\end{flushleft}
\begin{center}

{\Huge KIV/TI - Semestrální práce}
\\[0.3cm]
\vspace{1.7cm}
{\Large Kateřina Kratochvílová - A13B0364P}\\
\vspace{0.2cm}
{\normalsize dtwok8@students.zcu.cz}\\
\vspace{1cm}
{\Large Jan Kohlíček - A13B0350P}\\
\vspace{0.2cm}
{\normalsize kohl@students.zcu.cz }
\vfill
{\large \today}
\end{center}
\end{titlepage}

\tableofcontents
\thispagestyle{empty}



\chapter{Zadání}
\setcounter{page}{1}
Navrhněte konečněautomatový model pro řízení kotle na ohřev vody podle zadání:\\
\\
Po stisknutí tlačítka START obsluhou systému začne napouštět kotel a po dosažení určité minimální úrovně hladiny zapne topné spirály. Po dosažení maximální hladiny kotle přestane napouštět a dokončí ohřev. Po dosažení stanovené teploty dojde k vypnutí topných spirál. Předpokládáme kontinuální odběr teplé vody, kotel musí být schopen vodu dopouštět.\\
\\
Definujte potřebné vstupní a výstupní signály, automat popište přechodovým grafem.\\
\\
Model řídícího automatu realizujte softwarově na základě principů popsaných v materiálu. Všechny signály od čidel modelujte vstupy od klávesnice, řídicí signál a informaci o stavu vypisujte textově na obrazovku.\\


\chapter{Analýza úlohy}
Kotle bude přijímat impulsové signály od čidel hladinoměru a teploměru. Vždy bude pomoc přijmout jen jeden signál, na který může reagovat vysláním neomezeným počtem signálů.
Za těchto podmínek lze použít konečný automat Mealyho typu.\\
Stavy, kterými může kotel během celého cyklu projít: start, nečinnost, napouštění, topení, napouštění-topení,

Vstupní signály: 

Výstupní signály: 

\chapter{Implementace}
Simulace kotlu je konzolová aplikace, napsaná ve skriptovacím jazyce Python. Python byl zvolen pro jeho produktivnost z hlediska rychlosti psaní kódu.



\chapter{Uživatelská příručka}
\section{Spuštění aplikace}
Pro spuštění je potřeba mít nainstalovaný Python, který zle stáhnout z \url{https://www.python.org/downloads/}\\
Aplikaci spustíte ve složce projektu příkazem \texttt{"python boiler\_controller"}.\\
\\
Volitelné parametry:
\begin{itemize}
	\item \texttt{-h} ... vypíše nápovědu
	\item \texttt{-v} ... vypíše verzi
\end{itemize}


\chapter{Závěr}
V semestrální práci jsme vytvořili návrh automatu a jeho následnou implementaci.
Překvapilo nás, jak bylo obtížné a časově náročné navrhnout konečný automat, který by měl mít praktické použití. Tato zkušenost nám pomohla pochopit výhody a nevýhody konečných automatů.


\end{document}