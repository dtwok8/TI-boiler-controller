\documentclass[12pt]{report}
\usepackage[utf8]{inputenc}
\usepackage[IL2]{fontenc}
\usepackage[czech]{babel}
\usepackage{hyperref}
\usepackage{graphicx} 
\usepackage{float}
\title{Logické řízení - řízení kotle na ohřev vody}
\author{Jan Kohlíček}

\begin{document}

\begin{titlepage}
\begin{flushleft} 
{\includegraphics[width=.5\textwidth]{./images/fav_logo.jpg}\\[3cm]}
\end{flushleft}
\begin{center}

{\Huge KIV/TI - Semestrální práce\\[0.3cm]}\\
\vspace{1.7cm}
{\Large Kateřina Kratochvílová - A13B0364P}\\
\vspace{0.2cm}
{\normalsize dtwok8@students.zcu.cz}\\
\vspace{1cm}
{\Large Jan Kohlíček - A13B0350P}\\
\vspace{0.2cm}
{\normalsize kohl@students.zcu.cz }
\vfill
{\large \today}
\end{center}
\end{titlepage}

\tableofcontents
\thispagestyle{empty}



\chapter{Zadání}
\setcounter{page}{1}
Navrhněte konečněautomatový model pro řízení kotle na ohřev vody podle zadání:\\
\\
Po stisknutí tlačítka START obsluhou systému začne napouštět kotel a po dosažení určité minimální úrovně hladiny zapne topné spirály. Po dosažení maximální hladiny kotle přestane napouštět a dokončí ohřev. Po dosažení stanovené teploty dojde k vypnutí topných spirál. Předpokládáme kontinuální odběr teplé vody, kotel musí být schopen vodu dopouštět.\\
\\
Definujte potřebné vstupní a výstupní signály, automat popište přechodovým grafem.\\
\\
Model řídícího automatu realizujte softwarově na základě principů popsaných v materiálu. Všechny signály od čidel modelujte vstupy od klávesnice, řídicí signál a informaci o stavu vypisujte textově na obrazovku.\\


\chapter{Návrh}
\chapter{Implementace}
\chapter{Manuál}


\chapter{Závěr}


\end{document}